\documentclass[10pt, oneside]{article} 
\usepackage{amsmath, amsthm, amssymb, calrsfs, wasysym, verbatim, bbm, color, graphics, geometry, cite}
\geometry{tmargin=.75in, bmargin=.75in, lmargin=.75in, rmargin = .75in}  

\newcommand{\R}{\mathbb{R}}
\newcommand{\C}{\mathbb{C}}
\newcommand{\Z}{\mathbb{Z}}
\newcommand{\N}{\mathbb{N}}
\newcommand{\Q}{\mathbb{Q}}
\newcommand{\Cdot}{\boldsymbol{\cdot}}

\newtheorem{thm}{Theorem}
\newtheorem{defn}{Definition}
\newtheorem{conv}{Convention}
\newtheorem{rem}{Remark}
\newtheorem{lem}{Lemma}
\newtheorem{cor}{Corollary}
\newtheorem{example}{Example}
\newtheorem{exe}{Exercise}
\newtheorem{conjecture}{Conjecture}
\title{Schrodinger Trig Functions}
\author{[Drew Remmenga]}

\begin{document}

\maketitle
\begin{abstract}
    We take the perpendicular unit and use it to define a field with one unit. 
\end{abstract}
\section{Basic Properties of the F1}
    When we think about the field with one unit we can imagine several properties we would like that element to have. 
    Under the operation of addition we would like this unit to be its own inverse and as a group we would like it to be its own identity with that set.
    We write:
    \begin{align}
        \perp = -\perp \label{add:inv}
    \end{align}
    And we also would like if that following equation holds:
    \begin{align}
        \perp + \perp = \perp \label{add:id}
    \end{align}
    Now for multiplication. We would like it if in the field with one element the following equation holds:
    \begin{align}
        \perp * \perp^{-1} = \perp \label{mult:id}
    \end{align}
    We often fantasize that in a field with one unit this last equation holds:
    \begin{align}
        \perp * \perp = \perp \label{mult:inv}
    \end{align}
    Carlstrom introduced an element of a larger set which matches this description \cite{Carlstrom2001Wheels} and they called it the perpendicular unit or bottom unit which we shall adopt here.
    The properties of wheels with their unary operation / is not quite a multiplicative inverse. We extend it here in this single case. 
    We know by Wiel \cite{milne2015riemannhypothesisfinitefields} and Peak Math \cite{youtube:f1} the Riemann Hypothesis is true over finite fields and the field with one unit. 
\bibliographystyle{plain}  % or another style like alpha, unsrt, etc.
\bibliography{references.bib}  % the name of the .bib file
\end{document}

