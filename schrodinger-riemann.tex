\documentclass[10pt, oneside]{article}
\usepackage{amsmath, amsthm, amssymb, calrsfs, wasysym, verbatim, bbm, color, graphics, geometry, cite}
\geometry{tmargin=.75in, bmargin=.75in, lmargin=.75in, rmargin = .75in}




\newcommand{\R}{\mathbb{R}}
\newcommand{\C}{\mathbb{C}}
\newcommand{\Z}{\mathbb{Z}}
\newcommand{\N}{\mathbb{N}}
\newcommand{\Q}{\mathbb{Q}}
\newcommand{\Cdot}{\boldsymbol{\cdot}}




\newtheorem{thm}{Theorem}
\newtheorem{defn}{Definition}
\newtheorem{conv}{Convention}
\newtheorem{rem}{Remark}
\newtheorem{lem}{Lemma}
\newtheorem{cor}{Corollary}
\newtheorem{example}{Example}
\newtheorem{exe}{Exercise}
\newtheorem{conjecture}{Conjecture}
\newtheorem{remark}{Remark}
\title{Solutions to the Schrodinger Equation As a Ring}
\author{[Drew Remmenga]}




\begin{document}




\maketitle
\begin{abstract}
  Hilbert dreamed of a theory of Quantum gravity with eigenvalues given by the Riemann Zeta Function. We derive a quantum theory of spin and connect it to eigenvalues of L-Functions.
\end{abstract}
\section{Solving the Schrodinger Equation}
  Schroginger introduced his wave equation \cite{Islam1994} which in a field with zero potential we write:
  \begin{align}
      i \frac{\partial \psi (x,t)}{\partial t} = \frac{\partial^2 \psi (x,t)}{\partial x^2} \label{schrodinger}
  \end{align}
  This is idealized and set in natural units and in one dimension but works for our purposes.
  We begin with the separation of variables.
  \begin{align*}
      \psi (x,t) &= T(t)X(x) \\
      i T'(t) X(x) &= X''(x)T(t) \\
      i \frac{T'(t)}{T(t)} &= \frac{X''(x)}{X(x)} = \lambda
  \end{align*}
  Where $\lambda$ is some constant.
  Furthermore we write:
  \begin{align*}
      i T'(t) &= \lambda T(t) \\
      T(t) &= Ae^{i \lambda t}
  \end{align*}
  And:
  \begin{align*}
      X''(x) &= \lambda X(x) \\
      X(x) &= c_1 e^{\sqrt{\lambda} x} + c_2 e^{-\sqrt{\lambda} x}
  \end{align*}
  In idealized cases of the wave equation we fix the origin to be zero and a distance $L$ to be zero. For our purposes we shall set $L=\pi$.
  If we take $\lambda > 0$ we get the trivial solution. If we set $\lambda = 0$ we once again arrive at the trivial solution.
  We take $\lambda < 0$ and write:
  \begin{align*}
      0 = c_2 \sin (\mu_n x)
  \end{align*}
  Without loss of generality in the constant $c_2$. The constant $\mu_n = n \frac{\pi}{L}$.
  In the time dependent Schrodinger equation linear combinations of solutions are solutions so we need $n \in \N$ and $\lambda_n = -\mu_n^2$.
  We arrive at:
  \begin{align*}
      \psi (x,t) = \sum_{n=0}^{\infty} A_n e^{i t n^2} \sin(n x)
  \end{align*}
  Traditionally in quantum mechanics we renormalize the equation to find the missing constant $A_n$.
  Here we take the notion of fractional integrals. The fraction is given by the variable $\alpha$.
  \begin{align}
       _a D_t^{-\alpha} f(t) = \frac{1}{\Gamma (\alpha)} \int_{t}^{b} (\tau-t)^{\alpha-1} f(\tau) d\tau \label{Riemann-Liouville}
  \end{align}
   This \ref{Riemann-Liouville} is the Riemann-Liouville fractional integral \cite{Hermann2014}. Here we take $\alpha=s$ to be any complex number.
  \begin{align*}
      _a D_x^{-s} |\psi(x,t) |^2  &= 1 \\
      \frac{1}{\Gamma (s)} \int_{0}^{\pi} | (\pi - x)^{s-1} \psi(x,t) |^2 dx &= 1 \\
      \frac{1}{\Gamma (s)} \int_{0}^{\pi} | (\pi - x)^{s-1} \sum_{n=1}^{\infty} A_n \sin(nx) e^{i t n^2} |^2 dx &= 1 \\
      \frac{1}{\Gamma (s)} |\sum_{n=1}^{\infty} \frac{A_n}{2(-n)^{s}(-1)^{\frac{3s}{2}}} [-i (-1)^{n} (\Gamma(s, -i n(x-\pi))-(-1)^{s} \Gamma(s, i n (x-\pi)))]_{x=0}^{x=\pi} \psi(t)|^2 &= 1 \\
      \frac{1}{\Gamma (s)} |\sum_{n=1}^{\infty} \frac{A_n}{2(-n)^{s}(-1)^{\frac{3s}{2}}} [i (-1)^{n} (\Gamma(s, 0)) - (-1)^{s} \Gamma(s,0) - (-1) \Gamma(s,0) + i (-1)^{n} \Gamma (s,-i n \pi) + (-1)^{s} \Gamma(s, i n \pi) \psi(t)] |^{2} &= 1 \\   
   \end{align*}
   This is cool but we just want the Zeta function. We can absorb the constants in $\Gamma$ into $A_n$ without losing generality. As well as constants in $s$.
   \begin{align*}
       \sum_{n=1}^{\infty} \frac{A_n}{n^s} \frac{\bar{A_n}}{\bar{n^s}} e^{i n^2 t} \bar{e^{i n^2 t}} &= 1
   \end{align*}
   Now we take the natural log of both sides.
   \begin{align}
      \sum_{n=1}^{\infty} \frac{A_n \bar{A_n}}{n^{s}\bar{n^s}} e^{i t n^2} \bar{e^{i t n^2}} &=  1 \label{eq:1} \\
      ln[\sum_{n=1}^{\infty} \frac{A_n}{n^s} \frac{\bar{A_n}}{\bar{n^{s}}} e^{i t n^2} \bar{e^{i t n^2}}] &= 2 m \pi \label{eq:2}
  \end{align}
  \section{Proof Solutions form a Field}
  We shall take $m=0$ for our purposes.
  \begin{remark}
      This forms a ring.
  \end{remark}
  \begin{proof}
      Let $a$, $b$, and $c$ be solutions to \ref{eq:2}.
      Additions of these equations are associative.
      \begin{align*}
          (a+b)+c=a+(b+c)
      \end{align*}
      Multiplication is associative.
      \begin{align*}
          (a b) c = a (b c)
      \end{align*}
      Addition is commutative.
      \begin{align*}
          a + b = b + a
      \end{align*}
      Multiplication is commutative.
      \begin{align*}
          (a b) = (b a)
      \end{align*}
      Addition is distributive:
      \begin{align*}
          a (b + c) = ab + ac
      \end{align*}
      There exists a zero identity for addition. The trivial solution.
      There exists an identity for multiplication the $A_n$ which sets the wave equal to 1.
      There exists additive inverse elements. If $a$ is a solution to \ref{eq:1} then $-a$ is also a solution.
  \end{proof}
  \begin{remark}
      This forms a group isomorphic to $\mathbb{S^{\aleph_0}}$. This is the circle in countably infinite dimensions. Take an element of \ref{eq:2}. There exists a map from the $A_n$ terms to each dimension of the circle by renormalizing. This map is clearly bijective.
  \end{remark}
  \section{The Field of L-Functions}
  We know the Riemann Hypothesis over finite fields is true. \cite{milne2015riemannhypothesisfinitefields} \cite{youtube:f1} But more work must be done to prove it over the field of complex numbers.
  There exists a map between solutions of \ref{eq:1} to the set of L-Functions by setting $A_n$ appropriately and multiplying by a constant.
  Products of solutions to \ref{eq:1} are also solutions. We perform a product integral over the domain.
  This is reminiscent of what is described in the field with one unit here \cite{youtube:f1}. All solutions across time are solutions. We write:
  \begin{align*}
    \int_{-\infty}^{\infty}\sum_{n=1}^{\infty} \frac{A_n \bar{A_n}}{n^{s}\bar{n^s}} |e^{i t n^2} |^{2dt} &=  1
  \end{align*}
  $\psi(t)$ cycles through complex numbers in the circle. We can absorb without loss of generality into $A_n$ and $\bar{A_n}$ since we're finding the magnitude so this result about the Riemann Zeta function is stronger than the generalized Riemann Hypothesis.
  We can rewrite this as a contour integral through the entire complex plane.
  We can absorb $A_n$ and $\bar{A_n}$ into a new real number which we can call $\chi (n)$. We can also rewrite $s$ to $\frac{s}{2}$ without loss of generality since we are integrating over the entire complex plane. The general Riemann Hypothesis reads: \cite{Davenport2000}
  \begin{align}
      L(\chi,s) = \sum_{n=1}^{\infty} \frac{\chi(n)}{n^s} \label{general}
  \end{align}
  In our formulation we can once again integrate over the entire complex plane. In fact we can do this infinitely many times.
  \begin{align}
      \sum_{n=1}^{\infty} \frac{\chi(n)}{n^s} \frac{\bar{\chi(n)}}{\bar{n^s}} & = 1 \\
  \end{align}
  We obtain the Grand Riemann Zeta hypothesis as described here \cite{youtube:f1} and here \cite{Davenport2000}. This function clearly has eigenvalues at the negative even integers, the so called 'trivial solutions'. And assuming the variable $s$ isn't equal to one of those values it can be an eigenvalue if the real part of $s$ is equal to $\frac{1}{2}$.
  Suppose there exists a solution not on the real line or not on the critical line. This would fundamentally break the symmetry between subtraction and division demonstrated between the trivial and non-trivial zeros.
\bibliographystyle{plain}  % or another style like alpha, unsrt, etc.
\bibliography{references.bib}  % the name of the .bib file
\end{document}





